\section{Introduction}
This paper investigates and reports the development of a Cloud based Internet of Things (IoT) system extending self-monitoring for the ReRide project. The Bachelor project is carried out on the sixth semester of the "Software Development" BSc line on the IT-University of Copenhagen. The motivation for the project is an individual curiosity for IoT and its growing popularity and influence in IT and the everyday life. Exploring IoT can be done in many ways, but investigating it through an actual case provides a deeper relevance, thus making it a more attractive project. 

The following problem definition describes the scope of the project and the theoretical background needed for performing this study.

\subsection{Problem definition}
How can a Cloud based IoT system extend self-monitoring as part of the ReRide project?
\begin{itemize}
	\item What is "the Internet of Things"?
	\item What is a "Cloud based IoT system"?
\end{itemize}

\subsection{Case}
The study is based on a specific case introduced by the supervisor, who himself is part of the project. This section describes the case in order to give an overview of the context for this Bachelor project. \\

The ReRide project\cite{article:reride} investigates how \textit{digital technology for self-monitoring can be designed to help integrate physical rehabilitation with everyday activities}. The project explores this by designing self-monitoring in the context of motorbike riding. By using a belt with embedded flex sensor, back posture can be perceived by a "microprocessor controlled mechanically moving display unit", which can be mounted on the bike allowing the rider to adjust the back posture through feedback from the unit. The focus is on human-computer interaction (HCI) and the notion of \textit{embodied perception} in the design for rapid coupling between the rider's adjustments and the feedback provided.  \\

The design experiment was carried out in Bangalore, India, with a small group of bike riders suffering from back injuries. By integrating the above mentioned components to the motorbike and the rider, the data collected from the flex sensor would enable a servo motor to move the display unit placed on the motorbike's speedometer based on the posture of the rider. The idea was that when a rider is sitting in a correct posture the speedometer would become visible, but would block certain parts of the speedometer if the rider would change posture to a wrong position. The experiment concluded that the notion of rapid coupling mentioned above was promising, but the form of the display needed more work. 

