\section{Process}
\label{process}
The project is carried out following the Step Wise plan introduced in section \ref{method}, in which the next step (following step 3 for analyzing the project), is to \textit{Identify the products and activities}. This section will present the project plan including activities. An activity network\footnote{A model for scheduling activities and their relationships as a network, where time flows from left to right.} has been used to define the logical ordering of the system development, which is a product of the identified components required for implementing the system(REF Appendix). This is only seen as a guideline for where to begin and end for every iteration of the prototype development.\\

Step 5-8 in Step Wise are supposed to be carried out for every activity planned for a project, and finally when this is done the plan is ready to be executed. However, for this project these steps have not been done in-depth as their effect doesn't apply for this scope, as described in the following. 

\begin{itemize}
	\item The effort for carrying out each activity in the software development is difficult to asses for this project, as there is no previous work to base this on.
	\item The risks tied to each activity share the general risks identified as project characteristics in section \ref{proj_chars}.
	\item The resources needed to complete each activity does not make sense to define, as the development life cycle is based on evolutionary prototyping.
\end{itemize}

 This does not mean a plan for carrying out the activities is not needed, as there is a strict deadline for the projected to be handed in. For this, "Baseline planning"\footnote{Explain this technique, with REF to book} is used as the technique for creating a time line and defining requirements for deliverables and deadlines which can be seen in appendix REF. 
%Eventuelt beskriv hvorfor baseline planning bruges her

%Skal der her være et afsnit som beskriver det faktiske forløb?